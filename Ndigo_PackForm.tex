% SVN Info:
% $Date: 2019-06-04 12:35:07 +0200 (Di, 04 Jun 2019) $
% $Rev: 4976 $
% $Author: andreas $
\section{Memory Management}

The \textit{host buffer} is memory on the host's system in which the data recorded by the Ndigo5G is stored until it is acknowledged by the user.

The host buffer is managed by the DMA (direct memory access) driver. The DMA driver can only ever write to the host buffer if enough memory is free. That means, new packets will never overwrite old packets, unless they have been acknowledged.

If the host buffer is full, data may be lost. Then, the \texttt{CRONO\_PACKET\_FLAG\_HOST\_BUFFER\_FULL} bit of \texttt{crono\_packet::flags} is set. To ensure that this does not happen, the user must acknowledge packets fast enough by the analysis software. Note that data only has been lost due to a full host buffer if the \texttt{CRONO\_PACKET\_FLAG\_TRIGGER\_MISSED} bit of \texttt{crono\_packet::flags} is set.

\subsection{Acknowledge Packets}
A packet in the ring buffer will only be overwritten if it has been acknowledged. This can be done manually by the user by calling \texttt{ndigo\_acknowledge()} or automatically by the driver if in the call of \texttt{ndigo\_read()}, \texttt{acknowledge\_last\_read} of the \texttt{ndigo\_read\_in} structure \texttt{in} was set to \texttt{true} (see Section~\ref{cp:readout}).

Acknowledging a packet acknowledges all previous packets as well.

Be aware that acknowledging a packet \textit{immediately} invalidates it, and it is unsafe to attempt accessing its content.

\subsection{Ndigo5G-Internal Memory Layout}
The Ndigo5G uses internal FIFO (first-in, first-out) memories. In one of these FIFOs, referred to as the DMA FIFO, packets that are ready to be sent to the host system are buffered. If the DMA FIFO is full at any point, the affected packets \texttt{CRONO\_PACKET\_FLAG\_DMA\_FIFO\_FULL} bit of \texttt{crono\_packet::flags} is set. This does not mean that data has been necessarily lost. Only if the\newline\texttt{CRONO\_PACKET\_FLAG\_TRIGGER\_MISSED} bit is set has data been lost.


\section{Output Structure ndigo\tu packet}

    \cronvar{unsigned char}{channel}\\
    0 to 3 for the ADC input channels, 4 for the TDC, 5 for the timestamp channel.\par

    \cronvar{unsigned char}{card}\\
    Identifies the source card in case there are multiple boards present. Defaults to 0 if no value is assigned to the parameter \textsf{board\tu id} in Structure \textsf{ndigo\tu init\tu parameters} or set via\\ \textsf{int ndigo\tu set\tu board\tu id(ndigo\tu device *device, int board\tu id)}.\par

    \cronvar{unsigned char}{type}\\
    For the ADC channels this is set to 1 to signify 16-bit signed data.\par

    For the TDC channel it is set to 8 to signify 64-bit unsigned data.\par

    If the type field is 128 or greater than there is no data present, even if length is not 0. In these cases the length field may contain other data.\par

    \noindent
    \begin{small}
    \begin{tabular}{|c|c|p{9,5cm}|}
        \hline
        Type & Length Field & Description\\\hline
        \hline
        1 & Number of payload words & 16-bit signed samples from one of the ADCs\\\hline
        8 & Number of payload words & 64 Bit unsigned TDC Data, only for internal processing\\\hline
        128 & Bit pattern of trigger sources & Whenever at least one of the sources that is enabled for the timestamp channel triggers, one of these packets is generated. The length field contains the triggers active when this packet was created.\\\hline
    \end{tabular}
    \end{small}

    \cronvar{unsigned char}{flags}\par
    \indent\crondef{NDIGO\tu PACKET\tu FLAG\tu SHORTENED} 1\\
    \indent If the bit with weight 1 is set, the packet was truncated because the internal FIFO was full. Less than the requested number of samples have been written due to the full FIFO.\par
    \indent\crondef{NDIGO\tu PACKET\tu FLAG\tu PACKETS\tu LOST} 2\\
    \indent If the bit with weight 2 is set, there are lost triggers immediately preceding this packet due to insufficient host buffers. The DMA controller has discarded packets due to full host buffer.\par
    \indent\crondef{NDIGO\tu PACKET\tu FLAG\tu OVERFLOW} 4\\
    \indent If the bit with weight 4 is set, the packet contains ADC sample overflows.\par
    \indent\crondef{NDIGO\tu PACKET\tu FLAG\tu TRIGGER\tu MISSED} 8\\
    \indent If the bit with weight 8 is set, there are lost triggers immediately preceding this packet due to insufficient buffers. The trigger unit has discarded packets due to a full DMA FIFO or due to a full host buffer.\par
    \indent\crondef{NDIGO\tu PACKET\tu FLAG\tu DMA\tu FIFO\tu FULL} 16\\
    \indent If the bit with weight 16 is set, the internal DMA FIFO was full. Triggers only got lost if a subsequent package has the bit with weight 8 set. If this flag is set, the PCIe link speed of the host system may be too small.\par
    \indent\crondef{NDIGO\tu PACKET\tu FLAG\tu HOST\tu BUFFER\tu FULL} 32\\
    \indent If the bit with weight 32 is set, the host buffer was full. Triggers only got lost if a subsequent package has the bit with weight 8 set. If this flag is set, the user analysis software may be too slow.\par
    \indent\crondef{NDIGO\tu PACKET\tu FLAG\tu TDC\tu NO\tu EDGE} 64\\
    \indent If the bit with weight 64 is set, the packet from the TDC does not contain valid data and the timestamp is not corrected. No valid edge was found in TDC packet.\par

    \cronvar{unsigned int}{length}\\
    Number of 64-bit elements (each containing 4 samples) in the data array if type $< 128$.\par

    If type = 128 this is the pattern of trigger sources that where active in the clock cycle given by the timestamp. Bits are set according to the trigger sources, i.e. bit 0 is set if trigger A0 was active, bit 29 is set if trigger BUS3\tu PE was active. Use the \textsf{NDIGO\tu TRIGGER\tu SOURCE\tu *defines}{} to check for the bits set.\par

    \cronvar{unsigned \tu\tu int64}{timestamp}\\
    ADC channels A to D: Timestamp of the last word in the packet in picoseconds.\par

    TDC: Timestamp of the trigger event (falling edge) on the TDC channel in picoseconds.\par

    When \textsf{ndigo\tu process\tu tdc\tu packet()} is called once on the packet, the timestamp is replaced with the precise timestamp for the edge.\par

    Timestamp channel: Timestamp of the trigger event in ps.\par

    \cronvar{unsigned \tu\tu int64}{data[]}\\
    Sample data.\par
    For the Ndigo5G, each 64 bit word contains four 16-bit signed words from the ADC. The user can cast the array to \texttt{short*} to directly operate on the sample data.