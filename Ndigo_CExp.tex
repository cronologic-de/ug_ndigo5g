% SVN Info:
% $Date: 2019-06-04 12:35:07 +0200 (Di, 04 Jun 2019) $
% $Rev: 4976 $
% $Author: andreas $
\lstset{
	language=[Visual]C++,
	keywordstyle=\bfseries\sffamily\color[rgb]{0,0,1},
	identifierstyle=\sffamily,
	commentstyle=\color[rgb]{0.133,0.545,0.133},
	stringstyle=\sffamily\color[rgb]{0.627,0.126,0.941},
	showstringspaces=false,
	basicstyle=\small,
	numberstyle=\footnotesize,
	numbers=left,
	stepnumber=1,
	numbersep=10pt,
	tabsize=2,
	breaklines=true,
	prebreak = \raisebox{0ex}[0ex][0ex]{\ensuremath{\hookleftarrow}},
	breakatwhitespace=false,
	aboveskip={1.5\baselineskip},
  columns=fixed,
  upquote=true,
  extendedchars=true,
% frame=single,
% backgroundcolor=\color{lbcolor},
}
\begin{lstlisting}[frame=tlrb]
#include "Ndigo_interface.h"
#include <stdio.h>
#include <stdlib.h>

int main(int argc, char* argv[])
{
	ndigo_init_parameters params;
	ndigo_get_default_init_parameters(&params);
	
	params.card_index = 0;
	params.buffer_size[0] = 1<<23;
	params.drive_external_clock = true;
	params.is_slave = false;
	params.use_external_clock = false;
	
	int error_code;
	const char*error_message;
	ndigo_device* ndgo = ndigo_init(&params, &error_code, &error_message);
	if( error_code != NDIGO_OK ) {
		printf("\nError %d: %s\n", error_code, error_message);
		exit(-1);
	}

	ndigo_configuration config;
	ndigo_get_default_configuration(ndgo, &config);
	config.adc_mode = NDIGO_ADC_MODE_ABCD;
	
	// disable unused trigger blocks
	config.trigger_block[1].enabled = false;
	config.trigger_block[2].enabled = false;
	config.trigger_block[3].enabled = false;
	config.trigger_block[4].enabled = false;
	
	// configure trigger block 0
	config.trigger_block[0].enabled = true;
	config.trigger_block[0].minimum_free_packets = 1.0;
	config.trigger_block[0].precursor = 0;
	config.trigger_block[0].retrigger = 0;
	
	config.trigger_block[0].sources = NDIGO_TRIGGER_SOURCE_A0;
	config.trigger_block[0].length = 16;
	config.trigger_block[0].gates = NDIGO_TRIGGER_GATE_NONE;
	
	config.analog_offset[0] = 0.1;
	
	config.trigger[NDIGO_TRIGGER_A0].edge = true;
	config.trigger[NDIGO_TRIGGER_A0].rising = false;
	config.trigger[NDIGO_TRIGGER_A0].threshold = 0;

	if( ndigo_configure(ndgo, &config) != NDIGO_OK ) {
		printf("\nFatal configuration error. Aborting...\n");
		exit(-1);
	}
	
	ndigo_start_capture(ndgo);
	
	// counts the number of packets received
	int count = 0;
	
	while( count < 10 ) {
		ndigo_read_in in;
		// Do not wait for data
		// (if set to 1 the ndigo_acknowledge function has to be removed)
		in.acknowledge_last_read = 0;
		ndigo_read_out out;
		int result = ndigo_read(ndgo, &in, &out);
		if( !result ) {
			// buffer received with one or more packets
			ndigo_packet *packet = out.first_packet;
			while( packet <= out.last_packet ) {
				int length = 0;
				if( !(packet->type & NDIGO_PACKET_TYPE_TIMESTAMP_ONLY) )
					length = packet->length;
					
				printf("Card %02x, Channel %02x, Flags %02x, Length %6d, Timestamp %llu \n", packet->card, packet->channel, packet->flags, length, packet->timestamp);
				if( !(packet->type & NDIGO_PACKET_TYPE_TIMESTAMP_ONLY) ){
					short* data = (short*) packet->data;
					for( inti = 0; i < packet->length * 4; i++ )
						printf("%6d, ", *(data++));
					printf("\n\n");
				}
				// current packet pointer is invalid after call to ndigo_acknowledge
				ndigo_packet *next_packet = ndigo_next_packet(packet);
				ndigo_acknowledge(ndgo, packet);
				packet = next_packet;
				count++;
			}
		}
	}
	ndigo_close(ndgo);
	return0;
}

\end{lstlisting}